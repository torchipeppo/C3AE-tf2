%!TEX root = report.tex

\chapter{Datasets}

Our C3AE implementation was trained and tested on the following datasets:

\begin{itemize}
  \item \textbf{Wiki}:
    a large dataset containing 62,328 labelled
    images\footnote{Collected and avaliable at \url{https://data.vision.ee.ethz.ch/cvl/rrothe/imdb-wiki/}}
    collected from Wikipedia \cite{wiki}.
    Despite the size, it lacks samples of very young
    or very old people, and it is quite noisy.
    The cropped and aligned version of the dataset
    was used in order to ensure each picture has a single face in it.
    This dataset was used as a pre-training set for ((one experiment))
    and as the training set for the whole ablation study.
  \item \textbf{UTKFace}:
    a dataset containing over 20,000 labelled
    images\footnote{Collected and avaliable at \url{https://susanqq.github.io/UTKFace/}}.
    It covers a wider range of ages compared to Wiki,
    therefore this dataset was used as the main training set
    for ((that same experiment)) and as a fine-tuning set after
    the pre-training with Wiki.
  \item \textbf{FG-NET}:
    a dataset containing 1000 labelled
    images\footnote{Collected and avaliable at \url{https://yanweifu.github.io/FG_NET_data/}}
    \cite{fgnet}.
    It is significantly higher-quality and better-curated than the other
    datasets listed here, but it is extremely small as well.
    This dataset was used as the test set for all experiments.
\end{itemize}

\section{Preprocessing}
Each dataset initially came in the form of a set of image files each with the
corresponding ground-truth age encoded in the file name in some way.
In order to make the datasets usable by our model, the following
preprocessing procedure was applied to each of them.

First of all, the age information is extracted from the file name of each
image through a dataset-dependent regular expression. \\
Second, the face in each image is detected with the MTCNN face-detection
network\footnote{Imported as external code from \url{https://github.com/ipazc/mtcnn}}
\cite{mtcnn}, and the resulting information is used as a base to position
the three bounding boxes to be used later to generate the different crops
that C3AE uses as multi-scale context (see \fullref{chp:theorystuff}). \\
Then, faces with associated age outside the [0, 120] range are filtered
out of the dataset. Images where less or more than one face is detected
are also discarded.

Finally, images, age labels and bounding boxes are all organized into a
\texttt{pandas} table which is saved to disk in the \texttt{pickle} format,
in order to be loaded later by the other parts of the code.

\section{Data generation}
Training, validation and test data are generated in real time
during the respective phase from the preprocessed dataset. \\
This requires a small amount of additional processing, which normally
consists of applying a reflect-type padding to handle bounding boxes
that are partially outside the border of the image, then generating three
cropped images, one per bounding box, and finally resizing each of those crops
to 64×64 pixels, the accepted input size of the C3AE model.

\subsection{Training data augmentation}
Training data undergo additional transformations during this generation process
in order to augment the dataset and make the trained model more robust:

\begin{itemize}
  \item Random erasing: before adding the reflect padding, an arbitrary portion
    of the image is deleted and replaced with random noise \cite{random_erasing}.
  \item Random shift: Before cropping, each bounding box is independently
    shifted by a random amount. If any part of a box would move past the border
    of the padded image, it stops at the border instead.
  \item Random contrast, brightness and color temperature change\footnote{
      The temperature change code was imported as external from
      \url{https://www.askaswiss.com/2016/02/how-to-manipulate-color-temperature-opencv-python.html}
      and adapted to randomize the intensity of the change.
    }
  \item Random rotation
  \item Vertical flipping
\end{itemize}

Each operation has a random probability to be applied and is independent
of the others, and the intensity of each transformation (except flipping)
is also random, but limited.
