%!TEX root = report.tex

\chapter{Introduction}
In the past decade soft biometrics has emerged out to be a new area of interest for the researchers due to its growing real-world 
applications. This includes a classic learning problem in computer vision: the estimation of demographic traits, such as the age. 
Researchers are trying to develop models which can accurately estimate the age or the age group of a person using different 
biometric traits. Currently, neural networks give the best classification results for age estimation using human faces.
Many CNNs (convolutional neural networks) such as AlexNet, VggNet, GoogLeNet and ResNet are able to accomplish this task with promising 
performance.\

However, to obtain more precise accuracy these networks have grown deeper and larger. This trend has resulted in increasingly higher 
computational costs in either training or deploying. In particular, deploying the previously mentioned models on mobile phones, cars and 
robots is next to impossible due to the model size and computational cost.\\
Recently other models have been proposed with the aim to reduce the number of parameters, thus yielding lightweight models without 
weakening their efficience.

In this work we want to investigate the limits of compact models for small-scale images and focus on one the most compact models for age 
classification, implementing it in practice to evaluate its performance.\\
The following report presents the development of the final project for the Neural Networks course at Università degli studi di Roma 
"La Sapienza", A.Y. 2020/21.

\section{Related works}

Our work is based on the study made by Chao Zhang, Shuaicheng Liu, Xun Xu and Ce Zhu in the paper \textit{"C3AE: Exploring the Limits 
of Compact Model for Age Estimation"} \cite{c3ae} in which they propose a \textbf{C}ompact basic model, \textbf{C}ascaded training 
and multi-scale \textbf{C}ontext, aiming to tackle small-scale image \textbf{A}ge \textbf{E}stimation. The model is called \textbf{C3AE}.

The proposed model is able to achieve a state-of-the-art performance compared with alternative compact models and even outperforms many 
bulky models. With an extremely compact size of 0.25 MB for the full model, which is possibly the smallest that has been
obtained so far on the facial recognition, C3AE is suitable to be deployed even on low-end mobiles and embedded platforms.

%\section{Report organization}
%This report is structured as follows:
%\begin{itemize}
%    \item In Chapter 2 we describe the datasets used for the training and the evaluation of the model, as well as the preprocessing 
%      and the augmentation techniques used on them.
%    \item In Chapter 3 we present the theory behind C3AE, explaining its model composition and our implementation.
%    \item ... 
%  \end{itemize}